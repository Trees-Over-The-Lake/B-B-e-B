\begin{abstract}
    Para começar falando sobre os algoritmos de otimização de problemas,
    começarei falando sobre o que é um problema linear na programação.
    Se considerarmos que exista um problema $X$ que leva $Y$ de tempo
    para ser resolvido e que $Y$ cresce diretamente proporcinal ao crescimento
    de $X$, então temos um problema linear. A ideia de todos os algoritmos
    que serão citados nesse documento é fazer $Y$ crescer mais lentamente.
    Dessa forma, se o problema $X$ dobrar de tamanho $Y$ não dobrará necessariamente de
    tamanho junto. Nesse trabalho iremos apresentar cinco tipos de algotimos
    para atingir esse objetivo. Com foco no \emph{Branch and Bound} e no
    \emph{Backtracking} mostraremos seus conceitos separadamente. Assim como,
    suas semelhanças e diferenças com outros algoritmos que tentam resolver
    o mesmo problema, mas com abordagens diferentes.
\end{abstract}