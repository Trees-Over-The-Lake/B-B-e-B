\emph{Backtracking} é um algoritmo genérico para encontrar alguns
problemas computacionais, que vai incrementando os candidatos para a solução
e se em algum momento for encontrado que algum dos candidatos não resolve
o problema, então vai acontecer um ``\emph{backtracking}''. Um ``\emph{backtracking}''
é o processo de retornar e ignorar essa solução, uma vez que ela não é resolve
o problema \cite{BacktrackingAlgorithmsExplained}.

\section{Quando usar}

O backtracking é aplicado em alguns tipos de problemas específicos \cite{Backtracking}.
Ele é uma ferramenta importante para resolver problemas de satisfação de restrições, como palavras cruzadas, aritmética verbal, Sudoku e muitos outros quebra-cabeças. 
Frequentemente, é a técnica mais conveniente para análise, para o problema da mochila e outros problemas de otimização combinatória. 
Ele depende dos ``procedimentos de caixa preta'' fornecidos pelo usuário que definem o problema a ser resolvido, a natureza dos candidatos parciais e como eles são estendidos para candidatos completos. 
É, portanto, uma metaheurística em vez de um algoritmo específico $-$ embora, ao contrário de muitas outras metaheurísticas, seja garantido que encontrará todas as soluções para um problema finito em um período de tempo limitado \cite{backtracking-algorithms}. 

Para alguns casos, o \emph{backtracking} é usado para o problema de enumeração, a fim de encontrar o conjunto de todas as soluções viáveis para o problema.
Por outro lado, o \emph{backtracking} não é considerado uma técnica otimizada para resolver um problema. 
Ele encontra sua aplicação quando a solução necessária para um problema não tem limite de tempo.