\subsection{Conceito}

A ideia inicial da progamação dinâmica está baseada em armazenar todas
as operações que serão repetidas durante uma recursão. Dessa forma,
se durante a recursão tiver um nó igual que já foi previamente calculado,
não será necessário que o cálculo seja feito novamente. Em árvores criadas
como o algoritmo de fibonacci em que muitos nós são iguais, todos eles
serão calculados apenas uma vez \cite{DynamicProgramming}.

\subsection{Exemplo de código}

\lstinputlisting[language=python]{../src/progamacaoDinamica/fib.py}

\subsection{Complexidade}
Como o algoritmo armazena todas as equações já feitas, todas as operações
são transformar de $\bigO(n^2)$ para $\bigO(n)$. Precisando apenas do
tempo de encontrar o elemento dentro do dicionário do \emph{Python} e retornando
ele \cite{FibonnaciMemoization}.