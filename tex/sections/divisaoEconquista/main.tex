\subsection{Conceito}

Em ciência da computação, \emph{divide and conquer} é um paradigma de projeto de algoritmo. 
Um algoritmo de divisão e conquista divide recursivamente um problema em dois ou mais
subproblemas do mesmo tipo ou de tipo relacionado, até que se tornem simples o suficiente para serem resolvidos diretamente. 
As soluções para os subproblemas são então combinadas para fornecer uma solução para o 
problema original.

A técnica de dividir e conquistar é a base de algoritmos eficientes para muitos problemas, 
como classificação, multiplicação de grandes números, localização do par mais próximo de pontos e análise sintática.

\subsubsection*{Esta técnica pode ser dividida nas seguintes três partes:}

\begin{itemize}
    \item Dividir: Envolve a divisão do problema em subproblemas menores.
    \item Conquistar: Resolva subproblemas chamando recursivamente até que seja resolvido.
    \item Combinar: Combine os subproblemas para obter a solução final de todo o problema.
\end{itemize}

A grande relação entre o \emph{Divide and Conquer}, o \emph{Branch and Bound} e o \emph{Backtracking} é o fato de todos serem de 
natureza recursiva devido a maneira em que suas estrategias são conceptualizadas. 
O \emph{Backtracking} é uma otimização do \emph{brute-force}, onde o algoritmo retorna quando percebe que não tem como 
achar a resposta pelo caminho atual, por isso ele na maioria das vezes é recursivo. 
O divisão e conquista divide o problema em subproblemas, usando a recursão para esse processo. 
Eles possuem a diferença de: O \emph{backtracking} tenta fazer apenas parte da solução, em busca da solução ótima, 
já o \emph{divide and conquer} sempre vai dividir o problema inteiro para a sua resolução.

\subsubsection*{A seguir estão alguns algoritmos padrão que seguem o algoritmo \emph{Divide and Conquer}:}

\begin{itemize}
    \item Quicksort
    \item MergeSort
    \item Par de pontos mais próximo
\end{itemize}

\subsubsection*{Exemplo: Encontrar o elemento máximo e mínimo de um array:}

Entrada: {70, 250, 50, 80, 140, 12, 14}

Saída: o número mínimo em uma determinada matriz é: 12

O número máximo em uma determinada matriz é: 250

Abordagem: encontrar o elemento máximo e mínimo de um determinado array é um aplicativo para dividir para conquistar. 
Neste problema, encontraremos os elementos máximo e mínimo em um determinado array. 
Estamos usando uma abordagem de dividir e conquistar (\emph{DAC} \cite{DivideAndConquer}) que tem três etapas: dividir, conquistar e combinar.

Neste problema, estamos usando a abordagem recursiva para encontrar o máximo, onde veremos que 
apenas dois elementos sobraram e então podemos facilmente usar a condição, ou seja, $if (a [indice]> a [indice + 1])$.
Na condição acima, verificamos a condição do lado esquerdo para descobrir o máximo. Agora, veremos a 
condição do lado direito para encontrar o máximo.
Função recursiva para verificar o lado direito do índice atual de uma matriz.
Agora, vamos comparar a condição e verificar o lado direito do índice atual de um determinado array.
No programa fornecido, vamos implementar essa lógica para verificar a condição do lado direito no índice atual.

Para encontrar o mínimo, vamos implementar uma logica recursiva igual fizemos para achar o maximo. 

\subsection{Exemplo de algoritmo}
\lstinputlisting[language=python]{../src/divisaoEconquista/divideAndConquer.py}
\cite{DivideAndConquerAlgorithm}

\subsection{Complexidade}

A complexidade dos algoritmos de divisão e conquista, por conta de sua 
abordagem, podem ir de $\bigO{n^2}$ como um \emph{bubblesort}
para $n\log_2 n$ como um \emph{quicksort} \cite{DivideAndConquerComplexity}.