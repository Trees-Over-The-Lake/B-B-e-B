A principal semelhança entre ambos os algoritmos é tentar resolver problemas
mais complexos de formas mais simples, divindo em subconjuntos menores e
os resolvendo separadamente. Com isso, ambos são algoritmos inventados para otimizar problemas normalmente
combinatórios que exigiriam que fossem testados todos os casos para se obter a 
resposta. Se \emph{Backtracking} conseguir encontrar uma solução mais rápida
ele vai escolhê-la, se não descartá-la e não resolverá aquele subconjunto.
\emph{Branch and Bound} vai um pouco além, não se limitando apenas a tentar resolver
o problema, mas a resolver de forma mais eficiente. Então aqui encontramos a primeira
diferença. O primeiro algoritmo tentará resolver o problema de qualquer forma,
enquanto o segundo só resolverá se for mais rápido do que um simples \emph{bruteforce}.

\subsubsection*{Para ficar mais simples o entendimento:}

\begin{center}
    \begin{tabular}{| m{2cm} | m{4cm} | m{4cm} |}
        \hline
        Parâmetro & \emph{Backtracking} & \emph{Branch and Bound} \\
        \hline
        Abordagem & 
        Usado para encontrar todas as soluções possíveis 
        disponíveis para um problema. Quando ele percebe que fez uma escolha 
        errada, ele desfaz a última escolha, fazendo o backup dela. 
        Ele pesquisa a árvore do espaço de estados até encontrar uma solução 
        para o problema. & 
        Usado para resolver problemas de otimização. Quando ele percebe que 
        já tem uma solução ótima melhor para a qual a pré-solução o conduz, 
        ele abandona essa pré-solução. Ele pesquisa completamente a árvore 
        do espaço de estado para obter a solução ideal. \\
        \hline
        Função &
        Envolve a função de viabilidade. &
        Envolve uma função delimitadora. \\
        \hline
        Problemas &
        Usado para resolver o problema de decisão. &
        Usado para resolver o problema de otimização. \\
        \hline
        Busca &
        O estado da árvore atual e buscado até que a solução seja obtida. &
        A solução ideal pode estar presente em qualquer lugar na árvore do 
        espaço de estado, a árvore precisa ser pesquisada completamente. \\
        \hline
        Eficiência &
        Mais eficiente. &
        Menos eficiente. \\
        \hline
    \end{tabular}
\end{center}