\documentclass{article}
\usepackage[utf8]{inputenc}
\usepackage[brazil]{babel}
\usepackage{setspace}
\usepackage{mathtools}
\usepackage{pgfplots}
\usepackage{listings}
\usepackage{xcolor}
\usepackage{graphicx}
\usepackage{indentfirst}
\usepackage{import}
\usepackage{hyperref}
\usepackage{outline}

\DeclareMathAlphabet{\mathcal}{OMS}{cmsy}{m}{n}
\SetMathAlphabet{\mathcal}{bold}{OMS}{cmsy}{b}{n}

\newcommand{\bigO}{\mathcal{O}}

\DeclarePairedDelimiter\ceil{\lceil}{\rceil}
\DeclarePairedDelimiter\floor{\lfloor}{\rfloor}

\definecolor{codegreen}{rgb}{0,0.6,0}
\definecolor{codegray}{rgb}{0.5,0.5,0.5}
\definecolor{codepurple}{rgb}{0.58,0,0.82}
\definecolor{backcolour}{rgb}{0.95,0.95,0.92}

\lstset{
  literate={ö}{{\"o}}1	{ä}{{\"a}}1	{ü}{{\"u}}1
           {ç}{{\,c}}1	{ã}{{\~a}}1	{á}{{\'a}}1
           {é}{{\'e}}1	{í}{{\'i}}1	{ó}{{\'o}}1
           {ú}{{\'u}}1	{Á}{{\'A}}1	{É}{{\'E}}1
           {Í}{{\'I}}1  {Ó}{{\'O}}1 {Ú}{{\'U}}1
           {â}{{\^a}}1  {ê}{{\^e}}1 {õ}{{\~o}}1
}

\lstdefinestyle{codigo}{
    commentstyle=\color{codegreen},
    keywordstyle=\color{magenta},
    numberstyle=\tiny\color{codegray},
    stringstyle=\color{codepurple},
    basicstyle=\ttfamily\footnotesize,
    breakatwhitespace=false,         
    breaklines=true,                 
    captionpos=b,                    
    keepspaces=true,                 
    numbers=left,                    
    numbersep=5pt,                  
    showspaces=false,                
    showstringspaces=false,
    showtabs=false,                  
    tabsize=2
}

\lstset{style=codigo}

\setstretch{1.5}
\pgfplotsset{width=10cm,compat=1.9}

\title{Branch and Bound e Backtracking}
\author{Lucas Fonseca Saliba \and Lucas Santiago de Oliveira}
\date{Novembro de 2021}


\begin{document}
\maketitle
\vspace{1cm}
\tableofcontents
\newpage

\import{sections/introducao}{introducao.tex}

\section{\emph{Branch and Bound}}
\subsection{Conceito}
\import{sections/branchandbound}{conceito.tex}

\subsection{Quando usar}
\import{sections/branchandbound}{uso.tex}

\subsection{Exemplo}
\import{sections/branchandbound}{exemplo.tex}

\section{\emph{Backtracking}}
\subsection{Conceito}
\import{sections/backtracking}{conceito.tex}

\subsection{Quando usar}
\import{sections/backtracking}{uso.tex}

\subsection{Exemplo}
\import{sections/backtracking}{exemplo.tex}

\section{Relação entre \emph{Branch and Bound} e \emph{Backtracking}}
\import{sections/branchandBack}{semelhancasEdiferencas.tex}

\section{Relação com a Abordagem Gulosa}
\import{sections/gulosa}{conceito.tex}

\section{Relação com Programação Dinâmica}
\import{sections/progamacaoDinamica}{main.tex}

\section{Relação com Divisão e Conquista}
\import{sections/divisaoEconquista}{main.tex}

\import{sections/referencias}{referencias.tex}

\end{document}